\newpage
\section{Оформление иерархии вложенности разделов}

В данном разделе приводится пример иерархии вложенности по п.~2.1.6 ГОСТ~19.106 \cite{gost_19_106_Требования_к_программным_документам_выполненным_печатным_способом} (см. приложение 2).

Абзацный отступ должен составлять ${1,25}$~сантиметра. 

Заголовок раздела выравнивается по центру, после номера раздела ставится точка, название раздела пишется прописными буквами без точки в конце.

\subsection{Подраздел}
Подраздел начинается с абзацного отступа, нумеруется вложенной нумерацией, где первая цифра\mdash номер раздела, вторая\mdash номер подраздела. После каждой цифры ставится точка. Заголовок подраздела начинается с прописной буквы (остальные строчные). Точка в конце названия подраздела не ставится.

\subsection{Еще один подраздел}
Текст текст текст текст текст текст текст текст текст текст текст текст текст текст текст текст текст текст текст текст текст текст текст текст текст текст текст.

\subsubsection{Пункт}
Пункты и подпункты, точно так же, как и подразделы, начинаются с абзацного отступа, нумеруются вложенной нумерацией, где первая цифра\mdash номер раздела, вторая\mdash номер подраздела, третья\mdash номер пункта, четвертая\mdash номер подпункта. После каждой цифры ставится точка. Заголовок начинается с прописной буквы (остальные строчные). Точка в конце названия не ставится.

%
%\paragraph{Пункт 2} 
%Текст текст текст текст текст текст текст текст текст текст текст текст текст текст текст текст текст текст текст текст текст текст текст текст текст текст текст.

\paragraph{Подпункт} 
Текст текст текст текст текст текст текст текст текст текст текст текст текст текст текст текст текст текст текст текст текст текст текст текст текст текст текст.
%
%\subparagraph{Подпункт 2} 
%Текст текст текст текст текст текст текст текст текст текст текст текст текст текст текст текст текст текст текст текст текст текст текст текст текст текст текст.

\newpage
\subsection{Верстка иерархии разделов в \LaTeX}

Команды, отвечающие за вставку:

\begin{enumerate}
	\item раздела: \lstinline|\section{}|;
	\item подраздела: \lstinline|\subsection{}|;
	\item пункта: \lstinline|\subsubsection{}|;
	\item подпункта: \lstinline|\paragraph{}|.
\end{enumerate}

В фигурных скобках указывается название, например, раздела: \lstinline|\section{Введение}|. Название пишется с первой прописной буквы, остальные строчные\mdash в таком виде название будет включено шаблоном в содержание документа, при этом в тексте документа название будет автоматически преобразовано во все прописные буквы, как того требует ГОСТ~19. 

ГОСТ~19 в явном виде не требует, чтобы раздел обязательно начинался с новой страницы, но этого требует (не документировано нигде) наш нормоконтроль. Начать оформление текста с новой страницы можно командой~\lstinline|\newpage|.

Приведем пример оформления иерархии вложенности текущего раздела, представленную на предыдущей странице:

\begin{lstlisting}[caption=\raggedright{Верстка иерархии вложенности}, frame=single, numbers=none]
\newpage
\section{Оформление иерархии вложенности разделов}
В данном разделе приводится...

\subsection{Подраздел}
Подраздел начинается с абзацного отступа...

\subsection{Еще один подраздел}
Текст текст текст текст текст...

\subsubsection{Пункт}
Пункты и подпункты, точно так же, как и подразделы...

\paragraph{Подпункт} 
Текст текст текст текст текст...
\end{lstlisting}


\underline{Внимание!} Шаблон не предусматривает пропуска ступеней иерархии вложенности разделов/подразделов/и~т.д., поэтому нельзя сделать в явном виде пункты сразу внутри раздела, минуя этап подраздела\mdash в таком случае вместо пропущенной ступени иерархии будет стоять ноль в нумерации, что недопустимо. Ситуация, когда нужно пропустить одну из ступеней иерархии вложенности разделов, довольно редкая, поэтому подробно не будет рассматриваться.

\underline{Внимание!} Язык разметки \LaTeX{} будет автоматически нумеровать разделы, подразделы и т.д., поэтому от технического писателя не требуется заранее проставлять нумерацию, а лишь обозначит место в иерархии\mdash это значительно упрощает процесс написания и верстку документации, когда в процессе разработки некоторые части документа могут быть переставлены местами и/или добавлены/удалены\mdash после такой правки нумерация не <<поедет>> и все ссылки сохранятся.

Также шаблон не позволяет создавать ступени иерархии без заголовка, как разрешает ГОСТ~19.106, что иногда требуется. Выход из ситуации возможен локально перегрузкой команд: 

\begin{lstlisting}[caption=\raggedright{Оформление иерархии вложенности}, frame=single, numbers=none]
{
	\renewcommand\theenumi{\thesection.\arabic{enumi}} % Перегрузка \theenumi
	\renewcommand\labelenumi{\theenumi.} % Перегрузка \labelenumi
	\addtolength{\labelwidth}{-3mm} % Правка ширины 
	...
	% Вставка подразделов/пунктов/подпунктов командой перечисления:
	\begin{enumerate}
		\item ... 
	\end{enumerate}
}
\end{lstlisting}

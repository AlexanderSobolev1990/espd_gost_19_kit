\newpage
\section{Оформление перечислений}

В данном разделе приводится пример оформления перечислений по п.~2.1.6 ГОСТ~19.106 \cite{gost_19_106_Требования_к_программным_документам_выполненным_печатным_способом} .

\subsection{Пример одноуровнего нумерованного перечисления} 

Каждый пункт нумерованного перечисления начинается с порядкового номера с закрывающей скобкой. Текст перечисления начинается со строчной буквы, в конце предложения ставится символ <<;>>. В конце последнего пункта перечисления ставится точка. Если пункт перечисления содержит более одного предложения, то они разделяются точкой, а следующее предложение начинается с прописной буквы (хотя таких случаев следует, по возможности, избегать):

\begin{enumerate}
\item текст текст текст текст текст текст текст текст текст текст текст текст текст текст текст;
\item первое длинное предложение. Второе длинное предложение;
\item текст текст текст текст текст текст текст текст текст текст текст текст текст текст текст.
\end{enumerate}

\subsection{Пример одноуровнего ненумерованного перечисления} 

Возможны также ненумерованные перечисления, и тогда каждый пункт начинается с символа <<-->>:

\begin{enumerate}
\item[--] текст текст текст текст текст текст текст текст текст текст текст текст текст текст текст;
\item[--] текст текст текст текст текст текст текст текст текст текст текст текст текст текст текст;
\item[--] текст текст текст текст текст текст текст текст текст текст текст текст текст текст текст.
\end{enumerate}

\subsection{Пример вложенного перечисления} 

В случае, когда требуется вложенное перечисление, первый уровень делается нумерованным, а второй уровень\mdash ненумерованным. Третий уровень вложенности и далее\mdash не рекомендуются. Первая строка первого уровня вложенности начинается с абзацного отступа, последующие строки имеют ширину как и обычный текст. Первая строка второго уровня вложенности начинается с двух абзацных отступов, а последующие строки с одного:

\begin{enumerate}
\item текст текст текст текст текст текст текст текст текст текст текст текст текст текст текст;
\item текст текст текст текст текст текст текст текст текст текст текст текст текст текст текст;
	\begin{enumerate}
	\item текст текст текст текст текст текст текст текст текст текст текст текст текст текст текст;
	\item текст текст текст текст текст текст текст текст текст текст текст текст текст текст текст;
	\item текст текст текст текст текст текст текст текст текст текст текст текст текст текст текст;
	\end{enumerate}	
	\item текст текст текст текст текст текст текст текст текст текст текст текст текст текст текст.
\end{enumerate}

\subsection{Верстка перечислений в \LaTeX} 

Приведенный в предыдущем подразделе пример набирается командами:

\begin{lstlisting}[caption=\raggedright{Верстка перечислений}, frame=single, numbers=none]
\begin{enumerate}
\item текст текст текст текст текст текст текст текст текст текст текст текст текст текст текст;
\item текст текст текст текст текст текст текст текст текст текст текст текст текст текст текст;
	\begin{enumerate}
	\item текст текст текст текст текст текст текст текст текст текст текст текст текст текст текст;
	\item текст текст текст текст текст текст текст текст текст текст текст текст текст текст текст;
	\item текст текст текст текст текст текст текст текст текст текст текст текст текст текст текст;
	\end{enumerate}	
\item текст текст текст текст текст текст текст текст текст текст текст текст текст текст текст.
\end{enumerate}
\end{lstlisting}

Заметим, что нумерация в явном виде не указывается, \LaTeX{} сам пронумерует список.

Если требуется ненумерованное одноуровневое перечисление, то нужно вручную проставить символ \lstinline|[--]| после команды \lstinline|\item|:

\begin{lstlisting}[caption=\raggedright{Верстка перечислений}, frame=single, numbers=none]
\item[--] текст текст текст текст текст текст текст текст текст текст текст текст текст текст текст;
\item[--] текст текст текст текст текст текст текст текст текст текст текст текст текст текст текст;
\item[--] текст текст текст текст текст текст текст текст текст текст текст текст текст текст текст.
\end{lstlisting}


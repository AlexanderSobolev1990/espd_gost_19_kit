\newpage
\section{Оформление перечислений}

В данном разделе приводится пример оформления перечислений по п.~2.1.6 ГОСТ~19.106 \cite{gost19106} .

\subsection{Пример одноуровнего нумерованного перечисления} 

\begin{enumerate}
\item текст текст текст текст текст текст текст текст текст текст текст текст текст текст текст текст текст текст текст текст текст текст текст текст текст текст текст;
\item текст текст текст текст текст текст текст текст текст текст текст текст текст текст текст текст текст текст текст текст текст текст текст текст текст текст текст;
\item текст текст текст текст текст текст текст текст текст текст текст текст текст текст текст текст текст текст текст текст текст текст текст текст текст текст текст.
\end{enumerate}

\subsection{Пример одноуровнего ненумерованного перечисления} 
\begin{enumerate}
\item[--] текст текст текст текст текст текст текст текст текст текст текст текст текст текст текст текст текст текст текст текст текст текст текст текст текст текст текст;
\item[--] текст текст текст текст текст текст текст текст текст текст текст текст текст текст текст текст текст текст текст текст текст текст текст текст текст текст текст;
\item[--] текст текст текст текст текст текст текст текст текст текст текст текст текст текст текст текст текст текст текст текст текст текст текст текст текст текст текст.
\end{enumerate}

\subsection{Пример вложенного перечисления} 

При таком перечислении 1 уровень делается нумерованным, 2 уровень\mdash ненумерованным. Уровень 3 и далее\mdash не рекомендуются.

\begin{enumerate}
\item текст текст текст текст текст текст текст текст текст текст текст текст текст текст текст текст текст текст текст текст текст текст текст текст текст текст текст;
\item текст текст текст текст текст текст текст текст текст текст текст текст текст текст текст текст текст текст текст текст текст текст текст текст текст текст текст;	
\begin{enumerate}
	\item текст текст текст текст текст текст текст текст текст текст текст текст текст текст текст текст текст текст текст текст текст текст текст текст текст текст текст;
	\item текст текст текст текст текст текст текст текст текст текст текст текст текст текст текст текст текст текст текст текст текст текст текст текст текст текст текст;
	\item текст текст текст текст текст текст текст текст текст текст текст текст текст текст текст текст текст текст текст текст текст текст текст текст текст текст текст;
\end{enumerate}	
\item текст текст текст текст текст текст текст текст текст текст текст текст текст текст текст текст текст текст текст текст текст текст текст текст текст текст текст.
\end{enumerate}

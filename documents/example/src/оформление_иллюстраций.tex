\newpage
\section{Оформление иллюстраций}

В данном разделе приводится пример оформления иллюстраций по п.~2.3 ГОСТ~19.106 \cite{gost_19_106_Требования_к_программным_документам_выполненным_печатным_способом}. где указано, что \textbf{подпись любой иллюстрации оформляется ключевым словом} <<Рис.~1>>, \textbf{а ссылка оформляется как }<<см.~рис.~1>>. Однако наш нормоконтроль требует, чтобы под иллюстрацией было написано <<Рисунок>>. 

Иллюстрации, если их в документе более одной, нумеруют арабскими цифрами по ГОСТ в пределах всего документа, по правилам нормоконтроля в пределах каждого раздела.

\illustration[][Тестовое изображение <<Лена>> 1][0.5]{Lenna}[fig:лена1]
\illustration[][Тестовое изображение <<Лена>> 2][0.25]{Lenna}[fig:лена2]

В тексте документа возможно вставлять ссылки на иллюстрации, например так: см. \ref{fig:лена1} или см. \ref{fig:лена2}.

\newpage
Вставка двух изображений рядом не рекомендуется (хотя и возможна):

{
\centering
\begin{tabular}[c]{ m{0.5\textwidth} m{0.5\textwidth} }		
	{
		\begin{minipage}[t]{0.45\textwidth}
			\centering
			\illustration[][Тестовое изображение <<Лена>> 3][1.0]{Lenna}[fig:лена3]
		\end{minipage}
	} & {
		\begin{minipage}[t]{0.45\textwidth}
			\centering
			\illustration[][Тестовое изображение <<Лена>> 4][1.0]{Lenna}[fig:лена4]
		\end{minipage}
	} \\		
\end{tabular}
}

\subsection{Верстка иллюстраций в \LaTeX}

Вставка изображений \ref{fig:лена1} и \ref{fig:лена2} выше осуществляется командами:

\begin{lstlisting}[caption=\raggedright{Верстка иллюстраций друг под другом}, frame=single, numbers=none]
\illustration[][Тестовое изображение <<Лена>> 1][0.5]{Lenna}[fig:лена1]
\illustration[][Тестовое изображение <<Лена>> 2][0.25]{Lenna}[fig:лена2]
\end{lstlisting}
где [<<Тестовое изображение "Лена" 1>>]\mdash подпись рисунка, [0.5]\mdash коэффициент масштаба, {Lenna}\mdash имя файла, [fig:лена1]\mdash метка, c~помощью которой в дальнейшем можно ссылаться на этот рисунок командой \lstinline|\ref{fig:лена1}|.

\newpage
Вставка изображений \ref{fig:лена3} и \ref{fig:лена4} выше осуществляется следующим образом:

\begin{lstlisting}[caption=\raggedright{Верстка иллюстраций в ряд}, frame=single, numbers=none]
\centering
\begin{tabular}[c]{ m{0.5\textwidth} m{0.5\textwidth} }		
{
	\begin{minipage}[t]{0.45\textwidth}
	\centering
	\illustration[][Тестовое изображение <<Лена>> 3][1.0]{Lenna}[fig:лена3]
	\end{minipage}
} & {
	\begin{minipage}[t]{0.45\textwidth}
	\centering
	\illustration[][Тестовое изображение <<Лена>> 4][1.0]{Lenna}[fig:лена4]
	\end{minipage}
} \\		
\end{tabular}
\end{lstlisting}

\newpage
\section{Введение}

Прежде чем приступить к непосредственно приведению примеров оформления, стоит заострить внимание на некоторых моментах оформления:

\begin{enumerate}
	\item размеры левого штампа листа утверждения и титульной страницы документа должны соответствовать ГОСТ~2.104~\cite{gost_2_104_Основные_надписи}. Однако, размеры штампа <<Перв.~применение>> отличаются от указанных в~\cite{gost_2_104_Основные_надписи} и составляют: расстояние от основного штампа 60~мм, размеры самого поля <<Перв. применение>> 35 мм, в то время как по ГОСТ требуется расстояние от основного штампа 22~мм, размеры самого поля <<Перв.~применение>> 60 мм, но оставим это на совести нормоконтроля и исполнителя; 
	\item правила выполнения чертежей и рамок по ГОСТ требуют, чтобы линии рамок были жирными, однако наш нормоконтроль принимает тонкие, как они и настроены в шаблоне;
	\item шрифт текста в рамках титульного листа, листа утверждения, листа регистрации изменений должен быть ГОСТ тип А, однако нормоконтроль соглашается/требует(?), чтобы шрифт был такой же, как и в тексте всего документа (Times New Roman);
	\item документ <<Техническое задание>> относится к программным документам (см. ГОСТ~19.101~\cite{gost_19_101_Виды_программ_и_программных_документов}, раздел 2, таблица 1) и, следовательно, выполняется согласно нормам и правилам оформления ГОСТ 19, если не указано иное;
	\item лист регистрации изменений в ЕСПД выполняется по ГОСТ~19.604\cite{gost_19_604_Правила_внесения_изменений} (см. раздел 3)\mdash это раздел документа, то есть он обязательный;
	\item перечень сокращений идет \textbf{в конце документа} после приложений и перед перечнем использованных источников и листом регистрации изменений согласно ГОСТ~19.106\cite{gost_19_106_Требования_к_программным_документам_выполненным_печатным_способом} (см.~раздел 1, п.1.6). Поскольку в ГОСТ~19 нет явных указаний по оформлению перечня сокращений, то следовало бы оформлять по ГОСТ~2.105\cite{gost_2_105_Общие_требования_к_текстовым_документам}, который ссылается в п.п.6.1.2 на ГОСТ~7.32\cite{gost_7_32_Отчет_о_НИР}, где насчет сокращений сказано в п.6.15: <<Перечень сокращений, условных обозначений, символов, единиц физических величин и определений должен располагаться столбцом \textbf{без знаков препинания в конце строки}. Слева \textbf{без абзацного отступа} в алфавитном порядке приводятся сокращения, условные обозначения, символы, единицы физических величин, а справа через тире\mdash их детальная расшифровка>>. Однако наш нормоконтроль, ссылаясь на стандарт организации (стандарты организации имеют приоритет над ГОСТ\sdash ами)\cite{sto_ПМ0_000_068_2015} (см.~п.6.3.2), утверждает, что перечень сокращений является перечислением и, следовательно, нужно ставить в конце знак препинания <<точка с запятой>>\mdash <<;>>;
	\item сокращение слова <<смотри>> до <<см.>> допускается по ГОСТ~19.106\cite{gost_19_106_Требования_к_программным_документам_выполненным_печатным_способом} (см.~п.2.8.1) и ГОСТ~2.316\cite{gost_2_316_Правила_нанесения_надписей} (см. приложение А);
	\item перечень источников, использованных при разработке называется <<Перечень использованных источников>>, а не <<Литература>> или <<Библиография>> (хотя в ГОСТ~19.404\cite{gost_19_404_Пояснительная_записка} приводится пример названия <<Источники, использованные при разработке>>). Согласно ГОСТ~7.32\cite{gost_7_32_Отчет_о_НИР} (см.~п.6.16 и пример в приложении Д) перечень источников приводится с абзацного отступа. 	
\end{enumerate}	


\newpage
\section{Оформление формул}

В данном разделе приводится пример оформления формул по п.~2.4 ГОСТ~19.106 \cite{gost_19_106_Требования_к_программным_документам_выполненным_печатным_способом}.

\subsection{Простые примеры вставки формул}

\subsubsection{Формула в тексте и без порядкового номера}

Пример формулы, вставляемой в тексте без присвоения порядкового номера: формула квадратного многочлена: $f(x) = ax^2 + bx + c$, где $a$\ndash первый (старший) коэффициент, $b$\ndash второй (средний) коэффициент, $c$\ndash свободный член.

\subsubsection{Отдельная формула с порядковым номером}\label{subsubsec:Отдельная формула с порядковым номером}
Пример формулы, вставляемой отдельной строкой с присвоением порядкового номера:
\begin{align}
	x = y + z, \label{eq:формула 1}
\end{align}

\formulalistsec{	
\item[где] $x$\ndash сумма, результат сложения слагаемых $y$ и $z$;
\item[] $y$\ndash первое слагаемое;
\item[] $z$\ndash второе слагаемое.	
}

Ссылка на формулу оформляется в круглых скобках: см. формулу (\ref{eq:формула 1}).

Формулы, вставленные на отдельных строках (и, как правило, с последующей расшифровкой входящих в них компонентов), имеют сквозную нумерацию арабскими цифрами. Такие формулы выравниваются по центру страницы, после формул ставится запятая (если далее идет расшифровка переменных, входящих в нее) или точка (если расшифровка не требуется и формула является логическим окончанием предложения) или ничего не ставится (если расшифровки переменных также нет, а далее после формулы следует текст, по смыслу являющийся продолжением текста до момента вставки формулы). Пример приводится ниже.

Стандартная линейная дискретная оптимизационная задача, широко известная как <<задача о назначениях>> (частный случай транспортной задачи, задача Монжа\nbdash Канторовича) формулируется следующим образом:

обеспечить максимум
\begin{align}
\textnormal{max}{\left[ \sum_{k=1}^{K}\sum_{l=1}^{L} (C_{k,l} \times \psi_{k,l}) \right]} \textbf{\text{(нет знака препинания!)}} \label{eq:Задача_о_назначениях}
\end{align}	

при ограничениях: \textbf{(ставится двоеточие!)}
\begin{align}
\sum_{k=1}^{K} C_{k,l} = \sum_{l=1}^{L} C_{k,l} = 1 \intdelim~C_{k,l} \in \{0,~1\}, \label{eq:Задача_о_назначениях_ограничения}
\end{align}	

\formulalistsec{
	\item[где] $K$\ndash число <<работников>>;
	\item[] $L$\ndash число <<работ>>; 
	\item[] $C_{k,l}$\ndash полученное решение: массив, элемент которого содержит единицу, если $l$\sdash я работа назначена $k$\sdash му работнику, или ноль в противном случае.
}

Очень длинные формулы переносят на новую строку на знаках <<$+,-,\times$>>. \textbf{Недопустимо переносить формулы на новую строку знаке деления}.

Перед формулой двоеточие ставят только тогда, когда этого требует построение текста, предшествующего формуле~\cite{sto_ПМ0_000_068_2015}. В частности, характерны следующие примеры:

\begin{enumerate}
	\item формула для расчета имеет следующий вид \textbf{(двоеточие ставится)}:
	\begin{align}
		a = b + c;
	\end{align}	
	\item дальнешний расчет ведется по формуле (в конце стоит <<по формуле>>, \textbf{двоеточие НЕ ставится})
	\begin{align}
		d = e \times f.
	\end{align}		
\end{enumerate}

\newpage
\subsection{Верстка формул в \LaTeX}

Верстка формул, наряду с версткой алгоритмов в псевдокоде, является ключевым моментом, ради которого Дональд Кнут и создал язык \LaTeX.
Приведем примеры оформления различных случаев.

\subsubsection{Верстка простой формулы в тексте}

\begin{lstlisting}[caption=\raggedright{Простая формула в тексте}, frame=single, numbers=none]
...формула квадратного многочлена: $f(x) = ax^2 + bx + c$, где...
\end{lstlisting}

\subsubsection{Верстка отдельной формулы с порядковым номером}

Дополнительно присваивается метка, по которой в дальнейшем можно ссылаться на формулу по номеру. Приведем пример верстка формулы из п.\ref{subsubsec:Отдельная формула с порядковым номером}:

\begin{lstlisting}[caption=\raggedright{Отдельная формула с порядковым номером}, frame=single, numbers=none]
\begin{align}
x = y + z, \label{eq:формула 1}
\end{align}

\formulalistsec{	
\item[где] $x$\ndash сумма, результат сложения слагаемых $y$ и $z$;
\item[] $y$\ndash первое слагаемое;
\item[] $z$\ndash второе слагаемое.	
}

Ссылка на формулу оформляется в круглых скобках: см. формулу (\ref{eq:формула 1}).
\end{lstlisting}

\newpage
\subsubsection{Верстка различных символов и математических знаков}

\underline{Внимание!} При наборе формул для печати кириллицы необходимо набрать команду \lstinline|\text{текст на кириллице}|.

{\tabletextsize
	\begin{longtable}[c]{| >{\raggedright}m{50mm} | >{\raggedright}m{120mm} | }
		%----------------------- преамбула ---------------------	
		\caption{\normalsize Примеры верстки различных символов и математических знаков\hspace{25cm}}
		\label{t:примеры_верстка} \\
		\hline
		\centering{Символ/математический знак} & 
		\centering{Верстка \LaTeX} \tabularnewline
		\hhline{|=|=|}
		\endfirsthead % Конец заголовка на 1 странице
		\multicolumn{2}{l}{\textit{Продолжение таблицы \thetable}} \\ 
		\hline	
		% Способ, при котором раздельно формуруется выравнивание заголовка и контетнта
		\centering{Символ/метематический знак} & 
		\centering{Верстка \LaTeX} \tabularnewline
		\hhline{|=|=|}
		\endhead
		\hline
		%		\multicolumn{5}{r}{\tabletextsize см. далее}
		\endfoot
		\hline
		\endlastfoot	
		%------------------- табличные данные ------------------				%
		Числовой нижний индекс: $A_1$ & \lstinline|$A_1$| \tabularnewline\hline
		$A_{\text{нижний индекс}}$ & \lstinline|$A_{\text{нижний индекс}}$| \tabularnewline\hline
		$A^{\text{верхний индекс(степень)}}$ & \lstinline|$A^{\text{верхний индекс(степень)}}$| \tabularnewline\hline
		$A_{\text{нижний индекс}}^{\text{верхний индекс(степень)}}$ & \lstinline|$A_{\text{нижний индекс}}^{\text{верхний индекс(степень)}}$| \tabularnewline\hline
		Знак градуса: $\ldots\degree$ & \lstinline|$\ldots\degree$| \tabularnewline\hline
		Греческие буквы: & ~ \tabularnewline
		Альфа: $\alpha$ & \lstinline|$\alpha$| \tabularnewline
		Бета: $\beta$ & \lstinline|$\beta$| \tabularnewline
		Гамма: $\gamma$ & \lstinline|$\gamma$| \tabularnewline
		Дельта: $\delta, \Delta$ & \lstinline|$\delta, \Delta$| \tabularnewline
		Эпсилон: $\epsilon$ & \lstinline|$\epsilon$| \tabularnewline
		Зета: $\zeta$ & \lstinline|$\zeta$| \tabularnewline
		Эта: $\eta$ & \lstinline|$\eta$| \tabularnewline
		Тетта: $\theta$ & \lstinline|$\theta$| \tabularnewline
		Йота: $\iota$ & \lstinline|$\iota$| \tabularnewline
		Каппа: $\kappa$ & \lstinline|$\kappa$| \tabularnewline
		Лямбда: $\lambda, \Lambda$ & \lstinline|$\lambda, \Lambda$| \tabularnewline
		Мю: $\mu$ & \lstinline|$\mu$| \tabularnewline
		Ню: $\nu$ & \lstinline|$\nu$| \tabularnewline
		Кси: $\xi$ & \lstinline|$\xi$| \tabularnewline
		Пи: $\pi, \Pi$ & \lstinline|$\pi, \Pi$| \tabularnewline
		Ро: $\rho$ & \lstinline|$\rho$| \tabularnewline
		Сигма: $\sigma, \Sigma$ & \lstinline|$\sigma, \Sigma$| \tabularnewline		
		Тау: $\tau$ & \lstinline|$\tau$| \tabularnewline		
		Фи: $\phi$ & \lstinline|$\phi$| \tabularnewline		
		Хи: $\chi$ & \lstinline|$\chi$| \tabularnewline		
		Пси: $\psi, \Psi$ & \lstinline|$\psi, \Psi$| \tabularnewline				
		Омега: $\omega, \Omega$ & \lstinline|$\omega, \Omega$| \tabularnewline\hline		
		Матрица жирным шрифтом без наклона: $\matr{A}$ & \lstinline|$\matr{A}$| \tabularnewline\hline			
		Матрица нежирным шрифтом с наклоном: $A$ & \lstinline|$A$| \tabularnewline\hline			
		Верхние символы (крышки) над одиночными буквами в сочетаниях с индексами: $\hat{\alpha}_0, \hat{A}, \tilde{B}, \bar{C}, \check{D}, \vec{E}, \hat{\mu}_{kl}^i$ & \lstinline|$\hat{\alpha}_0, \hat{A}, \tilde{B}, \bar{C}, \check{D}, \vec{E}, \hat{\mu}_{kl}^i$| \tabularnewline\hline
		Верхние символы над многобуквенными обозначениями: $\widehat{AB}, \widetilde{CD}, \overline{ABC}, \overrightarrow{DEF}$ & \lstinline|$\widehat{AB}, \widetilde{CD}, \overline{ABC}, \overrightarrow{DEF}$| \tabularnewline\hline		
		Скобки: \[1\] & \lstinline|1| \tabularnewline\hline		
		Фигурные скобки (обозначение элемента вектора, например): $\sigma_{i}^{j}\{ x \}=1$ & \lstinline|$\sigma_{i}^{j}\{ x \}=1$| \tabularnewline\hline
		Умножение (крестом): $A = B \times C$ & \lstinline|$A = B \times C$| \tabularnewline\hline
		Деление (в одну строку): $A = B / C$ & \lstinline|$A = B / C$| \tabularnewline\hline
		Деление (дробь без выравнивания высоты): ${A = \frac{B}{C}}$ & \lstinline|${A = \frac{B}{C}}$| \tabularnewline\hline
		Деление (дробь) c выравниванием высоты: \[A = \frac{B}{C}\]& \lstinline|\[A = \frac{B}{C}\]| \tabularnewline\hline
		Многоэтажная дробь:\[X=\frac{\ln\left(\cfrac{A}{B}\right)} {\ln\left(\cfrac{C}{D} \right)}\] & \lstinline|\[X=\frac{\ln\left(\cfrac{A}{B}\right)} {\ln\left(\cfrac{C}{D} \right)}\]| \tabularnewline\hline
		Знак радикала: $\sqrt{a}, \sqrt[3]{b}$ & \lstinline|$\sqrt{a}, \sqrt[3]{b}$| \tabularnewline\hline
		Производная штрихами: $f', f'', f'''$ & \lstinline|$f', f'', f'''$| \tabularnewline\hline
		Производная точками: $\dot{f}, \ddot{f}, \dddot{f}$ & \lstinline|$\dot{f}, \ddot{f}, \dddot{f}$| \tabularnewline\hline		
		Производная и частная производная дробью: \[\frac{d f(x)}{d x}, \frac{\partial f(x)}{\partial x}\] & \lstinline|\[\frac{d f(x)}{d x}, \frac{\partial f(x)}{\partial x}\]| \tabularnewline\hline
		Интеграл c разным размещением пределов интегрирования: \[\int_0^{\infty}f(x)dx, \int\limits_a^{b}f(x)dx \] & \lstinline|\[\int_0^{\infty}f(x)dx, \int\limits_a^{b}f(x)dx \]| \tabularnewline\hline
		Знак суммы c разным размещением пределов суммирования: \[\sum_{i=1}^n a_i, \sum\nolimits_{i=1}^n b_i\] & \lstinline|\[\sum_{i=1}^n a_i, \sum\nolimits_{i=1}^n b_i\]| \tabularnewline\hline
		Знак произведения c разным размещением пределов перемножения: \[\prod_{i=1}^n a_i, \prod\nolimits_{i=1}^n b_i\] & \lstinline|\[\prod_{i=1}^n a_i, \prod\nolimits_{i=1}^n b_i\]| \tabularnewline\hline
	\end{longtable}
}




%%%%%%%%%%%%%%%%%%%%%%%%%%%%%%%%%%%%%%%%%%%%%%%%%%%%%%%%%%%%%%%%%%%%%%%%%%%%%%%%
%\item Сумма, умножение, деление, дробь:
%{\zerodisplayskips
%	\begin{align}
%	\text{сумма:}\quad & A+B=C, \label{eq:ф6}\\
%	\text{умножение:}\quad & A\times B=C, \label{eq:ф7}\\
%	\text{деление через косую черту:}\quad & A/B=C, \label{eq:ф8}\\
%	\text{дробь (решение квадратного уравнения):}\quad & x_{1,2}=\frac{-b\pm\sqrt{b^2-4ac}}{2a}, \label{eq:ф9}\\
%	\text{дробь (решение квадратного уравнения):}\quad & x_{1,2}=\frac{-b\pm\sqrt{b^2-4ac}}{2a}. \label{eq:ф99}
%	\end{align}
%}%
%
%Знак <<\&>> внутри формулы и конструкции \verb=\begin{align}...\end{align}= вызывает выравнивание по этому символу. 
%
%Обратите внимание, повторная вставка формулы (\ref{eq:ф99}) вызывает автоматическое выравнивание по высоте.
%%%%%%%%%%%%%%%%%%%%%%%%%%%%%%%%%%%%%%%%%%%%%%%%%%%%%%%%%%%%%%%%%%%%%%%%%%%%%%%
%\item Производная и интеграл:
%{\zerodisplayskips
%	\begin{equation}	
%	f'\quad f''\quad
%	\dot{f}\quad \ddot{f} \quad
%	\frac{d f}{d x}\quad
%	\frac{\partial f}{\partial x}
%	\int_0^{\infty}\quad
%	\int\limits_0^{\infty}.\quad
%	\label{eq:ф10}
%	\end{equation}
%}%
%%%%%%%%%%%%%%%%%%%%%%%%%%%%%%%%%%%%%%%%%%%%%%%%%%%%%%%%%%%%%%%%%%%%%%%%%%%%%%%
%\item Знак суммы:
%{\zerodisplayskips
%	\begin{equation}	
%	\sum_{i=1}^n a_i,\quad
%	\sum\nolimits_{i=1}^n b_i.
%	\label{eq:ф11}
%	\end{equation}
%}
%%%%%%%%%%%%%%%%%%%%%%%%%%%%%%%%%%%%%%%%%%%%%%%%%%%%%%%%%%%%%%%%%%%%%%%%%%%%%%%%
%\item Перенос формул вручную c указанием места разделения и команды \verb=split=:
%{\zerodisplayskips
%	\begin{equation}	
%	\begin{split}
%	x&=1000+1100+{}\\
%	 &+1200+1300.
%	\end{split}
%	\label{eq:ф12}
%	\end{equation}
%}
%%%%%%%%%%%%%%%%%%%%%%%%%%%%%%%%%%%%%%%%%%%%%%%%%%%%%%%%%%%%%%%%%%%%%%%%%%%%%%%%
%\item Cистема уравнений с фигурной скобкой (выравнивание по знаку <<=>>):
%{%\zerodisplayskips
%	\begin{equation}	
%	\left\{
%	\begin{aligned}
%	x^2+y^2&=7 \\
%	x+y & = 3. \\
%	\end{aligned}
%	\right.
%	\end{equation}
%}
%%%%%%%%%%%%%%%%%%%%%%%%%%%%%%%%%%%%%%%%%%%%%%%%%%%%%%%%%%%%%%%%%%%%%%%%%%%%%%%%
%\item Cистема уравнений с фигурной скобкой (выравнивание по левому краю):
%{%\zerodisplayskips
%	\begin{equation}	
%	\left\{
%	\begin{aligned}
%	& x^2+y^2=7 \\
%	& x+y=3.
%	\end{aligned}
%	\right.
%	\end{equation}
%}
%%%%%%%%%%%%%%%%%%%%%%%%%%%%%%%%%%%%%%%%%%%%%%%%%%%%%%%%%%%%%%%%%%%%%%%%%%%%%%%%
%\item Значение, зависящее от условий:
%{%\zerodisplayskips
%	\begin{equation}	
%	|\sin(x)|=
%	\begin{cases}
%	\sin(x), & 0<x<\pi, \\
%	-\sin(x), & \pi<x<2\pi.	
%	\end{cases}
%	\label{eq:ф13}
%	\end{equation}
%}
%%%%%%%%%%%%%%%%%%%%%%%%%%%%%%%%%%%%%%%%%%%%%%%%%%%%%%%%%%%%%%%%%%%%%%%%%%%%%%%
%\item Длина волны через частоту:
%{\zerodisplayskips
%	\begin{equation}	
%	\lambda=C/(Fr \times 10^3).
%	\end{equation}
%}
%%%%%%%%%%%%%%%%%%%%%%%%%%%%%%%%%%%%%%%%%%%%%%%%%%%%%%%%%%%%%%%%%%%%%%%%%%%%%%%
%\item Пример очень длинной формулы с переносом на 2 строки c выравниванием по знакам <<=>> и <<+>>:
%{%\zerodisplayskips
%	\begin{equation}
%	\begin{split}	
%	Vf_i&=X.V_i\times 0.5 \times (\cos((X.K_i - AzEndR_i)\times DgToRd) +{}\\
%	    &+\cos((X.K_i - AzEndT_i)\times DgToRd).
%	\label{eq:ф14}
%	\end{split}
%	\end{equation}
%}
%%%%%%%%%%%%%%%%%%%%%%%%%%%%%%%%%%%%%%%%%%%%%%%%%%%%%%%%%%%%%%%%%%%%%%%%%%%%%%%
%\item Еще пример очень длинной формулы с переносом на 3 строки c выравниванием по знакам <<=>> и <<+>>:
%{%\zerodisplayskips
%	\begin{equation}
%	\begin{split}	
%	ABCD&=X.V_i\times 0.5 \times (\cos((X.K_i - AzEndR_i)\times DgToRd) +{}\\
%	&+\cos((X.K_i - AzEndT_i)\times DgToRd)+{}\\
%	&+\cos((X.K_i - AzEndT_i)\times DgToRd).  
%	\label{eq:ф15}
%	\end{split}
%	\end{equation}
%}
%%%%%%%%%%%%%%%%%%%%%%%%%%%%%%%%%%%%%%%%%%%%%%%%%%%%%%%%%%%%%%%%%%%%%%%%%%%%%%%
%\item Пример автовыбора высоты скобок путем использования команд \verb=\left= и \verb=\right= соответственно:
%{\zerodisplayskips1324
%	\begin{equation}
%	f(x)=1+\left(\frac{1}{1-x^{2}}
%	\right)^3.
%	\end{equation}
%}
%%%%%%%%%%%%%%%%%%%%%%%%%%%%%%%%%%%%%%%%%%%%%%%%%%%%%%%%%%%%%%%%%%%%%%%%%%%%%%%
%\item Пример многоэтажной дроби:
%{%\zerodisplayskips
%	\begin{equation}
%	X=\frac{\ln\left(\cfrac{A}{B}\right)\times \ln\left(\cfrac{C}{D}\right)}{\ln\left(\cfrac{E}{F} \right)\times \ln\left(\cfrac{G}{H} \right)}.
%	\end{equation}
%}
%%%%%%%%%
%\end{enumerate}	
%


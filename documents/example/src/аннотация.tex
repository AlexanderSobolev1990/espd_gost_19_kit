\newpage\annotation

%Данный документ является примером оформления текста с использованием системы верстки (La)\TeX. Ссылка: \url{https://en.wikipedia.org/wiki/LaTeX}. Отличительной чертой проекта, намного повышающей удобство использования, является использование файла UseLatex.cmake, который позволяет легко и просто собирать исходные тексты из *.tex файлов путем написания соответствующего CMakeLists.txt (пример имеется в директории проекта) и вызова процесса сборки стандартным способом: \lstinline|mkdir build && cd build && cmake .. && make|.
%
%Доработанный класс espd.cls позволяет легко и просто оформлять титульную страницу и лист утверждения по ГОСТ-19, а также включает все необходимое оформление. Таким образом, использование данного класса и языка разметки (La)\TeX~позволяет техническому писателю сконцентрироваться на главном\mdash написании текста. Оформление формул, таблиц, вставка рисунков также значительно упрощаются, исключается их <<съезжание>>, как часто случается при исползовании текстового редактора Word, особенно разных версий.

В данном документе изложены наиболее часто встречающиеся конструкции, необходимые для написания технической документации на программные изделия по ГОСТ~19. Отдельное внимание уделено использованию языка \LaTeX{} для оформления документации. Для первичного ознакомления с \LaTeX{} рекомендуется \cite{LatexБалдин,LatexЛьвовский,LatexКузнецов}.

В разделе <<Введение>> приведены замечания по соответствию/не соответствию данного документа ГОСТу.

В разделе <<Оформление иерархии вложенности разделов>> приводится пример оформления разделов, подразделов, пунктов, подпунктов документа.


В документе приводится перечень использованных источников, где указаны государственные стандарты и стандарты организации, согласно которых приводится оформление текста в данном документе.

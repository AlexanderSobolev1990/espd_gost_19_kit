%%%%%%%%%%%%%%%%%%%%%%%%%%%%%%%%%%%%%%%%%%%%%%%%%%%%%%%%%%%%%%%%%%%%%%%%%%%%%%%%
%
% \file       preamble.tex
% \brief      Преамбула для проекта
% \details    Теоретически можно вынести настройку в класс ЕСПД
% \date       20.02.23 - создан
% \author     Соболев А.А.
%
%%%%%%%%%%%%%%%%%%%%%%%%%%%%%%%%%%%%%%%%%%%%%%%%%%%%%%%%%%%%%%%%%%%%%%%%%%%%%%%%
\usepackage{setspace}
\usepackage{fontspec}
\usepackage{cite} % Для цитирования
\usepackage{amsmath,bm}
\usepackage{gensymb} % Для знака градуса
\usepackage{mathtools}
\usepackage{listings} % Для листингов исходного кода
\usepackage{hhline} % Для двойной линии отчерка в заголовках таблиц
\usepackage{color} % Цвет для листингов
\usepackage{stackengine}
\usepackage{amssymb}
\usepackage{bigdelim}
%\usepackage{xstring}
%\usepackage[
%backend=biber,
%style=ext-authoryear%,
%%sorting=nyvt
%]{biblatex}
\usepackage{cite}

\usepackage[group-digits=false, output-decimal-marker={.}]{siunitx} % Для лёгкой замены разделителя целой и дробной частей (точка/запятая) писать \num{0.5} и тогда автоматом точка будет заменена на запятую, если выставить output-decimal-marker={.}

%\usepackage{tocstyle}
%\usetocstyle{standard}
%\settocfeature{raggedhook}{\raggedright}

%\usepackage{titlesec}
\usepackage{hyperref}
\hypersetup{
	pdfencoding=auto,
	colorlinks=false,
	linktoc=all,
	hypertexnames=false,
	hidelinks = true,
	bookmarks=true
%	bookmarksnumbered=true
}
\usepackage{bookmark}

%\usepackage{geometry}
%\geometry{
%	a4paper
%}

\newcommand{\intdelim}{,}%{;}%  разделитель интервала вида [0,360)
%%%%%%%%%%%%%%%%%%%%%%%%%%%%%%%%%%%%%%%%%%%%%%%%%%%%%%%%%%%%%%%%%%%%%%%%%%%%%%%%
% исправление чтобы пакет setspace мог менять межстрочный интервал!
\makeatletter
\let\@currsize\normalsize
\makeatother

%\bibliographystyle{ugost2008}
%\DeclareFieldFormat[article]{volume}{\bibstring{jourvol}\addnbspace #1}
%\DeclareFieldFormat[article]{number}{\bibstring{number}\addnbspace #1}
\bibliographystyle{ugost2008}
\setmainfont{Times New Roman}
%%%%%%%%%%%%%%%%%%%%%%%%%%%%%%%%%%%%%%%%%%%%%%%%%%%%%%%%%%%%%%%%%%%%%%%%%%%%%%%%
% Греческие буквы переопределение
\renewcommand{\kappa}{\varkappa}
\renewcommand{\epsilon}{\varepsilon}
\renewcommand{\phi}{\varphi}
%%%%%%%%%%%%%%%%%%%%%%%%%%%%%%%%%%%%%%%%%%%%%%%%%%%%%%%%%%%%%%%%%%%%%%%%%%%%%%%%
% Знаки >= <=
\renewcommand{\le}{\leqslant}
\renewcommand{\ge}{\geqslant}
%%%%%%%%%%%%%%%%%%%%%%%%%%%%%%%%%%%%%%%%%%%%%%%%%%%%%%%%%%%%%%%%%%%%%%%%%%%%%%%%
% Переопределение команд дефиса
%%%%%%%%%%%%%%%%%%%%%%%%%%%%%%%%%%%%%%%%%%%%%%%%%%%%%%%%%%%%%%%%%%%%%%%%%%%%%%%%
\newcommand{\sdash}{\nobreakdash-}  % Дефис неразрывный без пробелов до и после
\newcommand{\ndash}{\nobreakdash~--~}  % Короткое тире неразрывное с пробелами до и после
\newcommand{\mdash}{\nobreakdash~---~} % Длинное тире неразрывное с пробелами до и после
\newcommand{\nbdash}{\nobreakdash--}  % Короткое тире неразрывное без пробелов до и после
\newcommand{\mbdash}{\nobreakdash---} % Длинное тире неразрывное без пробелов до и после
%%%%%%%%%%%%%%%%%%%%%%%%%%%%%%%%%%%%%%%%%%%%%%%%%%%%%%%%%%%%%%%%%%%%%%%%%%%%%%%%
% Для таблиц longtable
%%%%%%%%%%%%%%%%%%%%%%%%%%%%%%%%%%%%%%%%%%%%%%%%%%%%%%%%%%%%%%%%%%%%%%%%%%%%%%%%
\setlength{\doublerulesep}{1.5pt}% Разделитель между двойным отчерком заголовка таблицы
\floatsetup[longtable]{LTcapwidth=table} % Выравнивание заголовка по левому краю
%%%%%%%%%%%%%%%%%%%%%%%%%%%%%%%%%%%%%%%%%%%%%%%%%%%%%%%%%%%%%%%%%%%%%%%%%%%%%%%%
% Определение форматирования исходного кода в тексте документа
%%%%%%%%%%%%%%%%%%%%%%%%%%%%%%%%%%%%%%%%%%%%%%%%%%%%%%%%%%%%%%%%%%%%%%%%%%%%%%%%
% Можно выделять исходники цветом, но зачем
%\definecolor{dkgreen}{rgb}{0,0.5,0}
%\definecolor{gray}{rgb}{0.5,0.5,0.5}
%\definecolor{mauve}{rgb}{0.58,0,0.82}
\lstset{
	frame=none,
	language=C++,
	aboveskip=3mm,
	belowskip=3mm,
	showstringspaces=false,
	columns=fixed, %flexible,
	basicstyle=\listingtextsize\fontspec{Noto Sans Mono},%{Courier New},
	extendedchars=true, % без символа \ !!! - так поддерживается русский язык в комментариях
	inputencoding=utf8x,
%    escapechar=|,
	numbers=left,
%	numberstyle=\tiny,
%	stepnumber=1,
%	numberstyle=\tiny\color{mauve},
%	keywordstyle=\color{blue},
%	commentstyle=\color{dkgreen},
%	stringstyle=\color{mauve},
	breaklines=true,
	breakatwhitespace=true,
	tabsize=3
}
%% Поддержка русского языка в комментариях к исходному коду
\makeatletter % see https://tex.stackexchange.com/a/320345
\lst@InputCatcodes
\def\lst@DefEC{%
	\lst@CCECUse \lst@ProcessLetter
	^^80^^81^^82^^83^^84^^85^^86^^87^^88^^89^^8a^^8b^^8c^^8d^^8e^^8f%
	^^90^^91^^92^^93^^94^^95^^96^^97^^98^^99^^9a^^9b^^9c^^9d^^9e^^9f%
	^^a0^^a1^^a2^^a3^^a4^^a5^^a6^^a7^^a8^^a9^^aa^^ab^^ac^^ad^^ae^^af%
	^^b0^^b1^^b2^^b3^^b4^^b5^^b6^^b7^^b8^^b9^^ba^^bb^^bc^^bd^^be^^bf%
	^^c0^^c1^^c2^^c3^^c4^^c5^^c6^^c7^^c8^^c9^^ca^^cb^^cc^^cd^^ce^^cf%
	^^d0^^d1^^d2^^d3^^d4^^d5^^d6^^d7^^d8^^d9^^da^^db^^dc^^dd^^de^^df%
	^^e0^^e1^^e2^^e3^^e4^^e5^^e6^^e7^^e8^^e9^^ea^^eb^^ec^^ed^^ee^^ef%
	^^f0^^f1^^f2^^f3^^f4^^f5^^f6^^f7^^f8^^f9^^fa^^fb^^fc^^fd^^fe^^ff%
	^^^^20ac^^^^0153^^^^0152%
	% Basic Cyrillic alphabet coverage
	^^^^0410^^^^0411^^^^0412^^^^0413^^^^0414^^^^0415^^^^0416^^^^0417%
	^^^^0418^^^^0419^^^^041a^^^^041b^^^^041c^^^^041d^^^^041e^^^^041f%
	^^^^0420^^^^0421^^^^0422^^^^0423^^^^0424^^^^0425^^^^0426^^^^0427%
	^^^^0428^^^^0429^^^^042a^^^^042b^^^^042c^^^^042d^^^^042e^^^^042f%
	^^^^0430^^^^0431^^^^0432^^^^0433^^^^0434^^^^0435^^^^0436^^^^0437%
	^^^^0438^^^^0439^^^^043a^^^^043b^^^^043c^^^^043d^^^^043e^^^^043f%
	^^^^0440^^^^0441^^^^0442^^^^0443^^^^0444^^^^0445^^^^0446^^^^0447%
	^^^^0448^^^^0449^^^^044a^^^^044b^^^^044c^^^^044d^^^^044e^^^^044f%
	^^^^0401^^^^0451^^^^0020%
	%%%
	^^00}
\lst@RestoreCatcodes
\makeatother
%%%%%%%%%%%%%%%%%%%%%%%%%%%%%%%%%%%%%%%%%%%%%%%%%%%%%%%%%%%%%%%%%%%%%%%%%%%%%%%%%
%% Задание перечислений буквами
%%%%%%%%%%%%%%%%%%%%%%%%%%%%%%%%%%%%%%%%%%%%%%%%%%%%%%%%%%%%%%%%%%%%%%%%%%%%%%%%%
%\usepackage{enumitem}
%
%\makeatletter
%\newcommand{\realasbuk}[1]{\expandafter\russian@realalph\csname c@#1\endcsname}
%
%\def\russian@realAlph#1{\ifcase#1\or
%	А\or Б\or В\or Г\or Д\or Е\or Ж\or
%	З\or И\or К\or Л\or М\or Н\or О\or
%	П\or Р\or С\or Т\or У\or Ф\or Х\or
%	Ц\or Ч\or Ш\or Щ\or Э\or Ю\or Я\else\xpg@ill@value{#1}{russian@Alph}\fi}
%\def\russian@realalph#1{\ifcase#1\or
%	а\or б\or в\or г\or д\or е\or ж\or
%	з\or и\or к\or л\or м\or н\or о\or
%	п\or р\or с\or т\or у\or ф\or х\or
%	ц\or ч\or ш\or щ\or э\or ю\or я\else\xpg@ill@value{#1}{russian@alph}\fi}
%
%\AddEnumerateCounter{\realasbuk}{\russian@realalph}{щ}
%\makeatother
%%%%%%%%%%%%%%%%%%%%%%%%%%%%%%%%%%%%%%%%%%%%%%%%%%%%%%%%%%%%%%%%%%%%%%%%%%%%%%%%
% Задание полей титульных страниц
%%%%%%%%%%%%%%%%%%%%%%%%%%%%%%%%%%%%%%%%%%%%%%%%%%%%%%%%%%%%%%%%%%%%%%%%%%%%%%%%
\newcommand{\productcodefull}{АБВГД.12345} % Код организации + Регистрационный код
\newcommand{\programname}{<<Галактический транклюкатор>>} % Название программы
\newcommand{\productname}{<<Шифр (название) изделия>>} % Шифр (название) изделия
\newcommand{\algoname}{<<Аббревиатура алгоритма>>} % Сокращенное названия алгоритма
\title{Изделие {\productname}\\Специальное программное обеспечение ПТУ\\Программный комплекс\\\programname} 
%%%%%%%%%%%%%%%%%%%%%%%%%%%%%%%%%%%%%%%%%%%%%%%%%%%%%%%%%%%%%%%%%%%%%%%%%%%%%%%%
% Отдельно для ИУЛ
\iulproductname{Изделие \productname\\Специальное программное обеспечение ПТУ\\Программный комплекс\\\programname}
\iulcontrolsumm{1234.5678.9012.3456}
\iulversion{111}
\iulnumoflastchange{222}
\iulnote{~}
\iuldeveloper{Разработчиков} % Разраб.
\iulcheck{Проверялов} % Проверил
\iulnormocontrol{Нормоконтролеров} % Н.контр
\iulapproved{Утверждаев} % Утв.
\firstapplication{АБВГД.12345} % Первое применение
\referencenumber{} % Справ №
%\organizationcode{01234} % Код организации
%\registrationcode{56789} % Регистрационный код
\productcode{\productcodefull} % Код организации + Регистрационный код
\redaction{01} % Номер редакции
\documentnumber{01} % Номер документа данного вида
%\partnumber{1} % Номер части документа - чтобы задать - надо фиксить класс espd
%%%%%%%%%%%%%%%%%%%%%%%%%%%%%%%%%%%%%%%%%%%%%%%%%%%%%%%%%%%%%%%%%%%%%%%%%%%%%%%%%
%
% Фамилии, должности и т.д. для титульников
%
% СОГЛАСОВАНО
\customerrank{Начальник\\межгалактической комиссии}
\customername{\phantom{А.Б.~Заказчиков}} % \phantom - убирает видимость

% УТВЕРЖДАЮ
\chiefconstructorrank{Главный конструктор\\изделия \productname}
\chiefconstructorname{А.Б.~Главный}

\fromcustomerrank{От межгалактической комиссии}
\fromcustomername{\phantom{А.Б.~Заказчиков}} % \phantom - убирает видимость

\headofdepartmentrank{Начальник Центра}
\headofdepartmentname{А.Б.~Чатланин}

\deputyofchiefconstructorrank{Зам.~гл.~конструктора}
\deputyofchiefconstructorname{А.Б.~Заместителев}
%
\developerrank{Разработчик}
\developername{А.Б.~Разработчиков}

%\headoflaboratoryrank{Начальник лаборатории 777}
%\headoflaboratoryname{А.Б.~Лабораториев}

% Исполнитель(и)
\authorname{А.Б.~Пацак}
%%
%\authori{А.Б.~Пацак1}
%\authorii{А.Б.~Пацак2}

\normocontrollerrank{Нормоконтроллер}
\normocontrollername{~}

\year{2023}
%%%%%%%%%%%%%%%%%%%%%%%%%%%%%%%%%%%%%%%%%%%%%%%%%%%%%%%%%%%%%%%%%%%%%%%%%%%%%%%%%
% Водяной знак
\usepackage{draftwatermark} % Использовать это! Не работает поверх рисунков, но их у нас мало 
\SetWatermarkText{PREPRINT}
\SetWatermarkLightness{0.8}
\SetWatermarkScale{0.8}
\SetWatermarkHorCenter{0.65\paperwidth}
\SetWatermarkVerCenter{0.55\paperheight}
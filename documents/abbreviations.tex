\renewcommand*{\aclabelfont}[1]{\acsfont{#1}} % Чтобы сокращения были не жирным шрифтом!

%\newcommand{\abbstart}{\acroextra{\hspace{1.0cm}}} % Начало строки - отступ
\newcommand{\abbstart}{} % Начало строки пустое (ГОСТ 7.32-2017)

%\newcommand{\abbdash}{\acroextra{\hspace{-3.5mm}--} } % Между сокращением и расшифровкой - тире (ГОСТ 7.32-2017)
\newcommand{\abbdash}{\acroextra{\hspace{-0.5mm}--} } % Между сокращением и расшифровкой - тире (ГОСТ 7.32-2017)
%\newcommand{\abbdash}{} % Между сокращением и расшифровкой - пусто

\newcommand{\abbendl}{\acroextra{;}} % Окончание строки - знак ";"
%\newcommand{\abbendl}{} % Окончание строки пустое (ГОСТ 7.32-2017)

%
% Сначала пишутся все русскоязычные акронимы, потом через пробел англоязычные
%

\begin{acronym}[\hspace{-0.2cm}] % В квадратных скобках указывается самое длинное сокращение для выравнивания!КК-РКФКБ
%\begin{acronym}[ЭВМ] % В квадратных скобках указывается самое длинное сокращение для выравнивания!
\acro{арм}[\abbstart АРМ]{\abbdash автоматизированное рабочее место\abbendl}
\acro{гт}[\abbstart ГТ]{\abbdash галактический транклюкатор\abbendl}
\acro{спо}[\abbstart СПО]{\abbdash специальное программное обеспечение\abbendl}
\acro{пту}[\abbstart ПТУ]{\abbdash пункт технического управления\abbendl}
\acro{эвм}[\abbstart ЭВМ]{\abbdash электронная вычислительная машина\abbendl}

\vspace{5mm}

\acro{utc}[\abbstart UTC]{\abbdash всемирное координированное время, англ.~coordinated universal time\abbendl}
\end{acronym}

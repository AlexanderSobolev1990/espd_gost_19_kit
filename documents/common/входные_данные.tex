Входные данные состоят из следующих структур данных:
\begin{enumerate}
\item[--] заголовок;
\item[--] входные данные;
\item[--] массив ка-цэ.
\end{enumerate}

Структуры данных приведены в таблицах ниже.

{\tabletextsize
\begin{longtable}[c]{| >{\raggedright}m{\wtname} | >{\centering}m{\wtsymbol} | >{\centering}m{\wtunits} | >{\centering}m{\wtbounds} | >{\centering}m{\wtcomment} |}
	%----------------------- преамбула ---------------------	
	\caption{\normalsize Структура заголовка\hspace{25cm}}
	\label{t:Входные_данные_заголовок_} \\
	\hline
	\centering{Наименование информации} & 
	\centering{Условное\\обозначение} & 
	\centering{Размер-\\ность} & 
	\centering{Пределы\\изменения} & 
	\centering{Примеча-\\ние} \tabularnewline
	\hhline{|=|=|=|=|=|}
	\endfirsthead % Конец заголовка на 1 странице
	\multicolumn{5}{l}{\textit{Продолжение таблицы \thetable}} \\ 
	\hline	
	% Способ, при котором раздельно формуруется выравнивание заголовка и контетнта
	\centering{Наименование информации} & 
	\centering{Условное\\обозначение} & 
	\centering{Размер-\\ность} & 
	\centering{Пределы\\изменения} & 
	\centering{Примеча-\\ние} \tabularnewline
	\hhline{|=|=|=|=|=|}
	\endhead
	\hline
	%		\multicolumn{5}{r}{\tabletextsize см. далее}
	\endfoot
	\hline
	\endlastfoot	
	%------------------- табличные данные ------------------		
	Контрольное слово & $CW$ & б/р & \lstinline|0xDEADBEEF| & uint32 \tabularnewline\hline
	Резерв & -- & -- & -- & uint32 \tabularnewline\hline
	Резерв & -- & -- & -- & uint32 \tabularnewline\hline
	Резерв & -- & -- & -- & uint32 \tabularnewline\hline
	Резерв & -- & -- & -- & uint32 \tabularnewline\hline
	Резерв & -- & -- & -- & uint32 \tabularnewline\hline
	Резерв & -- & -- & -- & uint32 \tabularnewline\hline
	Резерв & -- & -- & -- & uint32 \tabularnewline\hline
	Резерв & -- & -- & -- & uint32 \tabularnewline\hline
	Резерв & -- & -- & -- & uint32 \tabularnewline\hline
	Резерв & -- & -- & -- & uint32 \tabularnewline\hline
	Резерв & -- & -- & -- & uint32 \tabularnewline\hline
	Резерв & -- & -- & -- & uint32 \tabularnewline\hline
	Резерв & -- & -- & -- & uint32 \tabularnewline\hline
	Резерв & -- & -- & -- & uint32 \tabularnewline\hline
	Резерв & -- & -- & -- & uint32 \tabularnewline\hline
	Резерв & -- & -- & -- & uint32 \tabularnewline\hline
	Резерв & -- & -- & -- & uint32 \tabularnewline\hline
	Резерв & -- & -- & -- & uint32 \tabularnewline\hline
	Резерв & -- & -- & -- & uint32 \tabularnewline\hline
	Резерв & -- & -- & -- & uint32 \tabularnewline\hline
	Резерв & -- & -- & -- & uint32 \tabularnewline\hline
	Резерв & -- & -- & -- & uint32 \tabularnewline\hline
	Резерв & -- & -- & -- & uint32 \tabularnewline\hline
	Резерв & -- & -- & -- & uint32 \tabularnewline\hline
	Резерв & -- & -- & -- & uint32 \tabularnewline\hline
	Резерв & -- & -- & -- & uint32 \tabularnewline\hline
	Резерв & -- & -- & -- & uint32 \tabularnewline\hline
	Резерв & -- & -- & -- & uint32 \tabularnewline\hline
	Резерв & -- & -- & -- & uint32 \tabularnewline\hline
	Резерв & -- & -- & -- & uint32 \tabularnewline\hline
	Резерв & -- & -- & -- & uint32 \tabularnewline\hline
	Резерв & -- & -- & -- & uint32 \tabularnewline\hline
	Резерв & -- & -- & -- & uint32 \tabularnewline\hline
	Резерв & -- & -- & -- & uint32 \tabularnewline\hline
	Резерв & -- & -- & -- & uint32 \tabularnewline\hline
	Резерв & -- & -- & -- & uint32 \tabularnewline\hline
	Резерв & -- & -- & -- & uint32 \tabularnewline\hline
	Резерв & -- & -- & -- & uint32 \tabularnewline\hline
	Резерв & -- & -- & -- & uint32 \tabularnewline\hline
	Резерв & -- & -- & -- & uint32 \tabularnewline\hline
	Резерв & -- & -- & -- & uint32 \tabularnewline\hline
	Резерв & -- & -- & -- & uint32 \tabularnewline\hline
	Резерв & -- & -- & -- & uint32 \tabularnewline\hline
	Резерв & -- & -- & -- & uint32 \tabularnewline\hline
	Резерв & -- & -- & -- & uint32 \tabularnewline\hline
	Резерв & -- & -- & -- & uint32 \tabularnewline\hline
	Резерв & -- & -- & -- & uint32 \tabularnewline\hline
	Резерв & -- & -- & -- & uint32 \tabularnewline\hline
	Резерв & -- & -- & -- & uint32 \tabularnewline\hline
	Резерв & -- & -- & -- & uint32 \tabularnewline\hline
%	\multicolumn{5}{|l|}%
%	{%
%		\setstretch{0.7}%
%		%		\hspace{-1mm}% Добавление абзацного отступа (откуда он взялся - хз)
%		\tabletextsize%
%		\note Размер структуры N байт.
%	} \tabularnewline\hline
\end{longtable}
}

\newpage
{\tabletextsize
\begin{longtable}[c]{| >{\raggedright}m{\wtname} | >{\centering}m{\wtsymbol} | >{\centering}m{\wtunits} | >{\centering}m{\wtbounds} | >{\centering}m{\wtcomment} |}
	%----------------------- преамбула ---------------------	
	\caption{\normalsize Структура входных данных\hspace{25cm}}
	\label{t:Входные_данные} \\
	\hline
	\centering{Наименование информации} & 
	\centering{Условное\\обозначение} & 
	\centering{Размер-\\ность} & 
	\centering{Пределы\\изменения} & 
	\centering{Примеча-\\ние} \tabularnewline
	\hhline{|=|=|=|=|=|}
	\endfirsthead % Конец заголовка на 1 странице
	\multicolumn{5}{l}{\textit{Продолжение таблицы \thetable}} \\ 
	\hline	
	% Способ, при котором раздельно формуруется выравнивание заголовка и контетнта
	\centering{Наименование информации} & 
	\centering{Условное\\обозначение} & 
	\centering{Размер-\\ность} & 
	\centering{Пределы\\изменения} & 
	\centering{Примеча-\\ние} \tabularnewline
	\hhline{|=|=|=|=|=|}
	\endhead
	\hline
	%		\multicolumn{5}{r}{\tabletextsize см. далее}
	\endfoot
	\hline
	\endlastfoot	
	%------------------- табличные данные ------------------		
	Время приема пацаков & $Time$ & c & $0$\mbdash$(2^{32}-1)$ & uint32 \tabularnewline\hline	
	Резерв & -- & -- & -- & uint32 \tabularnewline\hline
	Резерв & -- & -- & -- & uint32 \tabularnewline\hline
	Резерв & -- & -- & -- & uint32 \tabularnewline\hline
	Резерв & -- & -- & -- & uint32 \tabularnewline\hline
	Резерв & -- & -- & -- & uint32 \tabularnewline\hline
	Резерв & -- & -- & -- & uint32 \tabularnewline\hline
	Резерв & -- & -- & -- & uint32 \tabularnewline\hline
	Резерв & -- & -- & -- & uint32 \tabularnewline\hline
	Резерв & -- & -- & -- & uint32 \tabularnewline\hline
	Резерв & -- & -- & -- & uint32 \tabularnewline\hline
	Резерв & -- & -- & -- & uint32 \tabularnewline\hline
	Резерв & -- & -- & -- & uint32 \tabularnewline\hline
	Резерв & -- & -- & -- & uint32 \tabularnewline\hline
	Резерв & -- & -- & -- & uint32 \tabularnewline\hline
	Резерв & -- & -- & -- & uint32 \tabularnewline\hline
	Резерв & -- & -- & -- & uint32 \tabularnewline\hline
	Резерв & -- & -- & -- & uint32 \tabularnewline\hline
	Резерв & -- & -- & -- & uint32 \tabularnewline\hline
	Резерв & -- & -- & -- & uint32 \tabularnewline\hline
	Резерв & -- & -- & -- & uint32 \tabularnewline\hline
	Резерв & -- & -- & -- & uint32 \tabularnewline\hline
	Резерв & -- & -- & -- & uint32 \tabularnewline\hline
	Резерв & -- & -- & -- & uint32 \tabularnewline\hline
	Резерв & -- & -- & -- & uint32 \tabularnewline\hline
	Резерв & -- & -- & -- & uint32 \tabularnewline\hline
	Резерв & -- & -- & -- & uint32 \tabularnewline\hline
	Резерв & -- & -- & -- & uint32 \tabularnewline\hline
	Резерв & -- & -- & -- & uint32 \tabularnewline\hline
	Резерв & -- & -- & -- & uint32 \tabularnewline\hline
	Резерв & -- & -- & -- & uint32 \tabularnewline\hline
	Резерв & -- & -- & -- & uint32 \tabularnewline\hline
	Резерв & -- & -- & -- & uint32 \tabularnewline\hline
	Резерв & -- & -- & -- & uint32 \tabularnewline\hline
	Резерв & -- & -- & -- & uint32 \tabularnewline\hline
	Резерв & -- & -- & -- & uint32 \tabularnewline\hline
	Резерв & -- & -- & -- & uint32 \tabularnewline\hline
	Резерв & -- & -- & -- & uint32 \tabularnewline\hline
	Резерв & -- & -- & -- & uint32 \tabularnewline\hline
	Резерв & -- & -- & -- & uint32 \tabularnewline\hline
	Резерв & -- & -- & -- & uint32 \tabularnewline\hline
	Резерв & -- & -- & -- & uint32 \tabularnewline\hline
	Резерв & -- & -- & -- & uint32 \tabularnewline\hline
	Резерв & -- & -- & -- & uint32 \tabularnewline\hline
	Резерв & -- & -- & -- & uint32 \tabularnewline\hline
	Резерв & -- & -- & -- & uint32 \tabularnewline\hline
	Резерв & -- & -- & -- & uint32 \tabularnewline\hline
	Резерв & -- & -- & -- & uint32 \tabularnewline\hline
	Резерв & -- & -- & -- & uint32 \tabularnewline\hline
	Резерв & -- & -- & -- & uint32 \tabularnewline\hline
	Резерв & -- & -- & -- & uint32 \tabularnewline\hline
	Резерв & -- & -- & -- & uint32 \tabularnewline\hline
	Резерв & -- & -- & -- & uint32 \tabularnewline\hline
	Резерв & -- & -- & -- & uint32 \tabularnewline\hline
	Резерв & -- & -- & -- & uint32 \tabularnewline\hline
	Резерв & -- & -- & -- & uint32 \tabularnewline\hline
%	\multicolumn{5}{|l|}%
%	{%
%		\setstretch{0.7}%
%		%		\hspace{-1mm}% Добавление абзацного отступа (откуда он взялся - хз)
%		\tabletextsize%
%		\note Размер структуры N байт.
%	} \tabularnewline\hline
\end{longtable}
}
%%%%%%%%%%%%%%%%%%%%%%%%%%%%%%%%%%%%%%%%%%%%%%%%%%%%%%%%%%%%%%%%%%%%%%%%%%%%%%%%
%
% Структура ка-цэ
%
{\tabletextsize
\begin{longtable}[c]{| >{\raggedright}m{\wtname} | >{\centering}m{\wtsymbol} | >{\centering}m{\wtunits} | >{\centering}m{\wtbounds} | >{\centering}m{\wtcomment} |}
	%----------------------- преамбула ---------------------	
	\caption{\normalsize Структура ка-цэ\hspace{25cm}}
	\label{t:Входные_данные_} \\
	\hline
	\centering{Наименование информации} & 
	\centering{Условное\\обозначение} & 
	\centering{Размер-\\ность} & 
	\centering{Пределы\\изменения} & 
	\centering{Примеча-\\ние} \tabularnewline
	\hhline{|=|=|=|=|=|}
	\endfirsthead % Конец заголовка на 1 странице
	\multicolumn{5}{l}{\textit{Продолжение таблицы \thetable}} \\ 
	\hline	
	% Способ, при котором раздельно формуруется выравнивание заголовка и контетнта
	\centering{Наименование информации} & 
	\centering{Условное\\обозначение} & 
	\centering{Размер-\\ность} & 
	\centering{Пределы\\изменения} & 
	\centering{Примеча-\\ние} \tabularnewline
	\hhline{|=|=|=|=|=|}
	\endhead
	\hline
	%		\multicolumn{5}{r}{\tabletextsize см. далее}
	\endfoot
	\hline
	\endlastfoot	
	%------------------- табличные данные ------------------		
	Ка-цэ & $ka-tce$ & мс & 0\mbdash 1000 & int32 \tabularnewline\hline	
	Резерв & -- & -- & -- & uint32 \tabularnewline\hline
	Резерв & -- & -- & -- & uint32 \tabularnewline\hline
	Резерв & -- & -- & -- & uint32 \tabularnewline\hline
	Резерв & -- & -- & -- & uint32 \tabularnewline\hline
	Резерв & -- & -- & -- & uint32 \tabularnewline\hline
	Резерв & -- & -- & -- & uint32 \tabularnewline\hline
	Резерв & -- & -- & -- & uint32 \tabularnewline\hline
	Резерв & -- & -- & -- & uint32 \tabularnewline\hline
	Резерв & -- & -- & -- & uint32 \tabularnewline\hline
	Резерв & -- & -- & -- & uint32 \tabularnewline\hline
	Резерв & -- & -- & -- & uint32 \tabularnewline\hline
	Резерв & -- & -- & -- & uint32 \tabularnewline\hline
	Резерв & -- & -- & -- & uint32 \tabularnewline\hline
	Резерв & -- & -- & -- & uint32 \tabularnewline\hline
	Резерв & -- & -- & -- & uint32 \tabularnewline\hline
	Резерв & -- & -- & -- & uint32 \tabularnewline\hline
	Резерв & -- & -- & -- & uint32 \tabularnewline\hline
	Резерв & -- & -- & -- & uint32 \tabularnewline\hline
	Резерв & -- & -- & -- & uint32 \tabularnewline\hline
	Резерв & -- & -- & -- & uint32 \tabularnewline\hline
	Резерв & -- & -- & -- & uint32 \tabularnewline\hline
	Резерв & -- & -- & -- & uint32 \tabularnewline\hline
	Резерв & -- & -- & -- & uint32 \tabularnewline\hline
	Резерв & -- & -- & -- & uint32 \tabularnewline\hline
	Резерв & -- & -- & -- & uint32 \tabularnewline\hline
	Резерв & -- & -- & -- & uint32 \tabularnewline\hline
	Резерв & -- & -- & -- & uint32 \tabularnewline\hline
	Резерв & -- & -- & -- & uint32 \tabularnewline\hline
	Резерв & -- & -- & -- & uint32 \tabularnewline\hline
	Резерв & -- & -- & -- & uint32 \tabularnewline\hline
	Резерв & -- & -- & -- & uint32 \tabularnewline\hline
	Резерв & -- & -- & -- & uint32 \tabularnewline\hline
	Резерв & -- & -- & -- & uint32 \tabularnewline\hline
	Резерв & -- & -- & -- & uint32 \tabularnewline\hline
	Резерв & -- & -- & -- & uint32 \tabularnewline\hline
	Резерв & -- & -- & -- & uint32 \tabularnewline\hline
	Резерв & -- & -- & -- & uint32 \tabularnewline\hline
	Резерв & -- & -- & -- & uint32 \tabularnewline\hline
	Резерв & -- & -- & -- & uint32 \tabularnewline\hline
	Резерв & -- & -- & -- & uint32 \tabularnewline\hline
	%
%	\multicolumn{5}{|l|}%
%	{%
%		\setstretch{0.7}%
%		%		\hspace{-1mm}% Добавление абзацного отступа (откуда он взялся - хз)
%		\tabletextsize%
%		\note Размер структуры N байт.
%	} \tabularnewline\hline
\end{longtable}
}
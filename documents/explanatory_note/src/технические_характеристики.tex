\newpage
\section{Технические характеристики}

\subsection{Постановка задачи на разработку программы}

Комплекс должен решать следующие задачи:

{
% хак для того, чтобы одно и то же содержние перечня задач отображалось здесь и в программе и методике испытания по разному
\renewcommand{\enumlineending}{;}
\renewcommand{\capitalisewords}{}
\begin{enumerate}
	\item\label{task:1} \capitalisewords{транклюкация} пацаков\enumlineending  
	\item\label{task:2} \capitalisewords{генерация} ка-цэ\enumlineending	
	{
	%%% --> Хак для выравнивания списка при использовании его в двух документах и по-разному пронумерованному
	\setlength{\labelsep}{2mm}%{5mm}% Зазор между "черточкой" списка (маркером) и текстом
	\setlength{\labelwidth}{-\parindent}
	\addtolength{\labelwidth}{-\labelsep}
	\addtolength{\labelwidth}{-4mm}
	\addtolength{\labelwidth}{-10mm}
%	\addtolength{\labelwidth}{2.5mm}
	%%% -->
	\begin{enumerate}
		\item с повышенной эффективностью;
		\item с обычной эффективностью\enumlineending	
	\end{enumerate}
	}	
\end{enumerate}
}

\subsection{Описание применяемых математических методов}

Рассматриваемый алгоритм реализует следующие основные методы и принципы:
\begin{enumerate}
	%
	\item[--] безопасная транклюкация пацаков;
	%
	\item[--] эффективная генерация ка-цэ.
\end{enumerate}

\newpage
\subsection{Описание алгоритма программы}

Алгоритм включает в себя основные этапы:
\begin{enumerate}
\item[--] транклюкация пацаков;
\item[--] генерация ка-цэ. 
\end{enumerate}

На этапе транклюкация все пацаки эффективно транклюкируются методом эффективной транклюкации.

На этапе генерации ка-цэ обеспечивается эффективная генерация ка-цэ для обогащения чатлан.

Все это безобразие успешно написано так, как указано в литературе\ifthenelse{\boolean{citemode}}{\cite{LatexБалдин}}{}.


%\newpage
\subsubsection{Структурная схема}
Структурная схема алгоритма представлена на \ref{fig:СхемаАлгоритма} и состоит из следующих блоков:
\begin{enumerate}
\item[--] блок 1 <<Транклюкация пацаков>>;
\item[--] блок 2 <<Генерация ка-цэ>>.
\end{enumerate}

Блок 1 <<Транклюкация пацаков>> осуществляет эффективную транклюкацию пацаков. 

Блок 2 <<Генерация ка-цэ>> производит эффективную генерацию ка-цэ. 

\illustration[][Структурная схема алгоритма][0.5]{algoscheme}[fig:СхемаАлгоритма]


\subsubsection{Формульно-логическое описание блоков алгоритма}

Бла - бла - бла

\newpage
\subsection{Взаимодействие программы с другими программами} 
Комплекс сопряжен по входу с комплексом <<Получение Пацаков и Ка-Цэ>>, от которого получает соответствующую входную информацию, а по выходу c потребителями ка-цэ.

\subsection{Входные и выходные данные}

Входные данные состоят из следующих структур данных:
\begin{enumerate}
\item[--] заголовок;
\item[--] входные данные;
\item[--] массив ка-цэ.
\end{enumerate}

Структуры данных приведены в таблицах ниже.

{\tabletextsize
\begin{longtable}[c]{| >{\raggedright}m{\wtname} | >{\centering}m{\wtsymbol} | >{\centering}m{\wtunits} | >{\centering}m{\wtbounds} | >{\centering}m{\wtcomment} |}
	%----------------------- преамбула ---------------------	
	\caption{\normalsize Структура заголовка\hspace{25cm}}
	\label{t:Входные_данные_заголовок_} \\
	\hline
	\centering{Наименование информации} & 
	\centering{Условное\\обозначение} & 
	\centering{Размер-\\ность} & 
	\centering{Пределы\\изменения} & 
	\centering{Примеча-\\ние} \tabularnewline
	\hhline{|=|=|=|=|=|}
	\endfirsthead % Конец заголовка на 1 странице
	\multicolumn{5}{l}{\textit{Продолжение таблицы \thetable}} \\ 
	\hline	
	% Способ, при котором раздельно формуруется выравнивание заголовка и контетнта
	\centering{Наименование информации} & 
	\centering{Условное\\обозначение} & 
	\centering{Размер-\\ность} & 
	\centering{Пределы\\изменения} & 
	\centering{Примеча-\\ние} \tabularnewline
	\hhline{|=|=|=|=|=|}
	\endhead
	\hline
	%		\multicolumn{5}{r}{\tabletextsize см. далее}
	\endfoot
	\hline
	\endlastfoot	
	%------------------- табличные данные ------------------		
	Контрольное слово & $CW$ & б/р & \lstinline|0xDEADBEEF| & uint32 \tabularnewline\hline
	Резерв & -- & -- & -- & uint32 \tabularnewline\hline
	Резерв & -- & -- & -- & uint32 \tabularnewline\hline
	Резерв & -- & -- & -- & uint32 \tabularnewline\hline
	Резерв & -- & -- & -- & uint32 \tabularnewline\hline
	Резерв & -- & -- & -- & uint32 \tabularnewline\hline
	Резерв & -- & -- & -- & uint32 \tabularnewline\hline
	Резерв & -- & -- & -- & uint32 \tabularnewline\hline
	Резерв & -- & -- & -- & uint32 \tabularnewline\hline
	Резерв & -- & -- & -- & uint32 \tabularnewline\hline
	Резерв & -- & -- & -- & uint32 \tabularnewline\hline
	Резерв & -- & -- & -- & uint32 \tabularnewline\hline
	Резерв & -- & -- & -- & uint32 \tabularnewline\hline
	Резерв & -- & -- & -- & uint32 \tabularnewline\hline
	Резерв & -- & -- & -- & uint32 \tabularnewline\hline
	Резерв & -- & -- & -- & uint32 \tabularnewline\hline
	Резерв & -- & -- & -- & uint32 \tabularnewline\hline
	Резерв & -- & -- & -- & uint32 \tabularnewline\hline
	Резерв & -- & -- & -- & uint32 \tabularnewline\hline
	Резерв & -- & -- & -- & uint32 \tabularnewline\hline
	Резерв & -- & -- & -- & uint32 \tabularnewline\hline
	Резерв & -- & -- & -- & uint32 \tabularnewline\hline
	Резерв & -- & -- & -- & uint32 \tabularnewline\hline
	Резерв & -- & -- & -- & uint32 \tabularnewline\hline
	Резерв & -- & -- & -- & uint32 \tabularnewline\hline
	Резерв & -- & -- & -- & uint32 \tabularnewline\hline
	Резерв & -- & -- & -- & uint32 \tabularnewline\hline
	Резерв & -- & -- & -- & uint32 \tabularnewline\hline
	Резерв & -- & -- & -- & uint32 \tabularnewline\hline
	Резерв & -- & -- & -- & uint32 \tabularnewline\hline
	Резерв & -- & -- & -- & uint32 \tabularnewline\hline
	Резерв & -- & -- & -- & uint32 \tabularnewline\hline
	Резерв & -- & -- & -- & uint32 \tabularnewline\hline
	Резерв & -- & -- & -- & uint32 \tabularnewline\hline
	Резерв & -- & -- & -- & uint32 \tabularnewline\hline
	Резерв & -- & -- & -- & uint32 \tabularnewline\hline
	Резерв & -- & -- & -- & uint32 \tabularnewline\hline
	Резерв & -- & -- & -- & uint32 \tabularnewline\hline
	Резерв & -- & -- & -- & uint32 \tabularnewline\hline
	Резерв & -- & -- & -- & uint32 \tabularnewline\hline
	Резерв & -- & -- & -- & uint32 \tabularnewline\hline
	Резерв & -- & -- & -- & uint32 \tabularnewline\hline
	Резерв & -- & -- & -- & uint32 \tabularnewline\hline
	Резерв & -- & -- & -- & uint32 \tabularnewline\hline
	Резерв & -- & -- & -- & uint32 \tabularnewline\hline
	Резерв & -- & -- & -- & uint32 \tabularnewline\hline
	Резерв & -- & -- & -- & uint32 \tabularnewline\hline
	Резерв & -- & -- & -- & uint32 \tabularnewline\hline
	Резерв & -- & -- & -- & uint32 \tabularnewline\hline
	Резерв & -- & -- & -- & uint32 \tabularnewline\hline
	Резерв & -- & -- & -- & uint32 \tabularnewline\hline
%	\multicolumn{5}{|l|}%
%	{%
%		\setstretch{0.7}%
%		%		\hspace{-1mm}% Добавление абзацного отступа (откуда он взялся - хз)
%		\tabletextsize%
%		\note Размер структуры N байт.
%	} \tabularnewline\hline
\end{longtable}
}

\newpage
{\tabletextsize
\begin{longtable}[c]{| >{\raggedright}m{\wtname} | >{\centering}m{\wtsymbol} | >{\centering}m{\wtunits} | >{\centering}m{\wtbounds} | >{\centering}m{\wtcomment} |}
	%----------------------- преамбула ---------------------	
	\caption{\normalsize Структура входных данных\hspace{25cm}}
	\label{t:Входные_данные} \\
	\hline
	\centering{Наименование информации} & 
	\centering{Условное\\обозначение} & 
	\centering{Размер-\\ность} & 
	\centering{Пределы\\изменения} & 
	\centering{Примеча-\\ние} \tabularnewline
	\hhline{|=|=|=|=|=|}
	\endfirsthead % Конец заголовка на 1 странице
	\multicolumn{5}{l}{\textit{Продолжение таблицы \thetable}} \\ 
	\hline	
	% Способ, при котором раздельно формуруется выравнивание заголовка и контетнта
	\centering{Наименование информации} & 
	\centering{Условное\\обозначение} & 
	\centering{Размер-\\ность} & 
	\centering{Пределы\\изменения} & 
	\centering{Примеча-\\ние} \tabularnewline
	\hhline{|=|=|=|=|=|}
	\endhead
	\hline
	%		\multicolumn{5}{r}{\tabletextsize см. далее}
	\endfoot
	\hline
	\endlastfoot	
	%------------------- табличные данные ------------------		
	Время приема пацаков & $Time$ & c & $0$\mbdash$(2^{32}-1)$ & uint32 \tabularnewline\hline	
	Резерв & -- & -- & -- & uint32 \tabularnewline\hline
	Резерв & -- & -- & -- & uint32 \tabularnewline\hline
	Резерв & -- & -- & -- & uint32 \tabularnewline\hline
	Резерв & -- & -- & -- & uint32 \tabularnewline\hline
	Резерв & -- & -- & -- & uint32 \tabularnewline\hline
	Резерв & -- & -- & -- & uint32 \tabularnewline\hline
	Резерв & -- & -- & -- & uint32 \tabularnewline\hline
	Резерв & -- & -- & -- & uint32 \tabularnewline\hline
	Резерв & -- & -- & -- & uint32 \tabularnewline\hline
	Резерв & -- & -- & -- & uint32 \tabularnewline\hline
	Резерв & -- & -- & -- & uint32 \tabularnewline\hline
	Резерв & -- & -- & -- & uint32 \tabularnewline\hline
	Резерв & -- & -- & -- & uint32 \tabularnewline\hline
	Резерв & -- & -- & -- & uint32 \tabularnewline\hline
	Резерв & -- & -- & -- & uint32 \tabularnewline\hline
	Резерв & -- & -- & -- & uint32 \tabularnewline\hline
	Резерв & -- & -- & -- & uint32 \tabularnewline\hline
	Резерв & -- & -- & -- & uint32 \tabularnewline\hline
	Резерв & -- & -- & -- & uint32 \tabularnewline\hline
	Резерв & -- & -- & -- & uint32 \tabularnewline\hline
	Резерв & -- & -- & -- & uint32 \tabularnewline\hline
	Резерв & -- & -- & -- & uint32 \tabularnewline\hline
	Резерв & -- & -- & -- & uint32 \tabularnewline\hline
	Резерв & -- & -- & -- & uint32 \tabularnewline\hline
	Резерв & -- & -- & -- & uint32 \tabularnewline\hline
	Резерв & -- & -- & -- & uint32 \tabularnewline\hline
	Резерв & -- & -- & -- & uint32 \tabularnewline\hline
	Резерв & -- & -- & -- & uint32 \tabularnewline\hline
	Резерв & -- & -- & -- & uint32 \tabularnewline\hline
	Резерв & -- & -- & -- & uint32 \tabularnewline\hline
	Резерв & -- & -- & -- & uint32 \tabularnewline\hline
	Резерв & -- & -- & -- & uint32 \tabularnewline\hline
	Резерв & -- & -- & -- & uint32 \tabularnewline\hline
	Резерв & -- & -- & -- & uint32 \tabularnewline\hline
	Резерв & -- & -- & -- & uint32 \tabularnewline\hline
	Резерв & -- & -- & -- & uint32 \tabularnewline\hline
	Резерв & -- & -- & -- & uint32 \tabularnewline\hline
	Резерв & -- & -- & -- & uint32 \tabularnewline\hline
	Резерв & -- & -- & -- & uint32 \tabularnewline\hline
	Резерв & -- & -- & -- & uint32 \tabularnewline\hline
	Резерв & -- & -- & -- & uint32 \tabularnewline\hline
	Резерв & -- & -- & -- & uint32 \tabularnewline\hline
	Резерв & -- & -- & -- & uint32 \tabularnewline\hline
	Резерв & -- & -- & -- & uint32 \tabularnewline\hline
	Резерв & -- & -- & -- & uint32 \tabularnewline\hline
	Резерв & -- & -- & -- & uint32 \tabularnewline\hline
	Резерв & -- & -- & -- & uint32 \tabularnewline\hline
	Резерв & -- & -- & -- & uint32 \tabularnewline\hline
	Резерв & -- & -- & -- & uint32 \tabularnewline\hline
	Резерв & -- & -- & -- & uint32 \tabularnewline\hline
	Резерв & -- & -- & -- & uint32 \tabularnewline\hline
	Резерв & -- & -- & -- & uint32 \tabularnewline\hline
	Резерв & -- & -- & -- & uint32 \tabularnewline\hline
	Резерв & -- & -- & -- & uint32 \tabularnewline\hline
	Резерв & -- & -- & -- & uint32 \tabularnewline\hline
%	\multicolumn{5}{|l|}%
%	{%
%		\setstretch{0.7}%
%		%		\hspace{-1mm}% Добавление абзацного отступа (откуда он взялся - хз)
%		\tabletextsize%
%		\note Размер структуры N байт.
%	} \tabularnewline\hline
\end{longtable}
}
%%%%%%%%%%%%%%%%%%%%%%%%%%%%%%%%%%%%%%%%%%%%%%%%%%%%%%%%%%%%%%%%%%%%%%%%%%%%%%%%
%
% Структура ка-цэ
%
{\tabletextsize
\begin{longtable}[c]{| >{\raggedright}m{\wtname} | >{\centering}m{\wtsymbol} | >{\centering}m{\wtunits} | >{\centering}m{\wtbounds} | >{\centering}m{\wtcomment} |}
	%----------------------- преамбула ---------------------	
	\caption{\normalsize Структура ка-цэ\hspace{25cm}}
	\label{t:Входные_данные_} \\
	\hline
	\centering{Наименование информации} & 
	\centering{Условное\\обозначение} & 
	\centering{Размер-\\ность} & 
	\centering{Пределы\\изменения} & 
	\centering{Примеча-\\ние} \tabularnewline
	\hhline{|=|=|=|=|=|}
	\endfirsthead % Конец заголовка на 1 странице
	\multicolumn{5}{l}{\textit{Продолжение таблицы \thetable}} \\ 
	\hline	
	% Способ, при котором раздельно формуруется выравнивание заголовка и контетнта
	\centering{Наименование информации} & 
	\centering{Условное\\обозначение} & 
	\centering{Размер-\\ность} & 
	\centering{Пределы\\изменения} & 
	\centering{Примеча-\\ние} \tabularnewline
	\hhline{|=|=|=|=|=|}
	\endhead
	\hline
	%		\multicolumn{5}{r}{\tabletextsize см. далее}
	\endfoot
	\hline
	\endlastfoot	
	%------------------- табличные данные ------------------		
	Ка-цэ & $ka-tce$ & мс & 0\mbdash 1000 & int32 \tabularnewline\hline	
	Резерв & -- & -- & -- & uint32 \tabularnewline\hline
	Резерв & -- & -- & -- & uint32 \tabularnewline\hline
	Резерв & -- & -- & -- & uint32 \tabularnewline\hline
	Резерв & -- & -- & -- & uint32 \tabularnewline\hline
	Резерв & -- & -- & -- & uint32 \tabularnewline\hline
	Резерв & -- & -- & -- & uint32 \tabularnewline\hline
	Резерв & -- & -- & -- & uint32 \tabularnewline\hline
	Резерв & -- & -- & -- & uint32 \tabularnewline\hline
	Резерв & -- & -- & -- & uint32 \tabularnewline\hline
	Резерв & -- & -- & -- & uint32 \tabularnewline\hline
	Резерв & -- & -- & -- & uint32 \tabularnewline\hline
	Резерв & -- & -- & -- & uint32 \tabularnewline\hline
	Резерв & -- & -- & -- & uint32 \tabularnewline\hline
	Резерв & -- & -- & -- & uint32 \tabularnewline\hline
	Резерв & -- & -- & -- & uint32 \tabularnewline\hline
	Резерв & -- & -- & -- & uint32 \tabularnewline\hline
	Резерв & -- & -- & -- & uint32 \tabularnewline\hline
	Резерв & -- & -- & -- & uint32 \tabularnewline\hline
	Резерв & -- & -- & -- & uint32 \tabularnewline\hline
	Резерв & -- & -- & -- & uint32 \tabularnewline\hline
	Резерв & -- & -- & -- & uint32 \tabularnewline\hline
	Резерв & -- & -- & -- & uint32 \tabularnewline\hline
	Резерв & -- & -- & -- & uint32 \tabularnewline\hline
	Резерв & -- & -- & -- & uint32 \tabularnewline\hline
	Резерв & -- & -- & -- & uint32 \tabularnewline\hline
	Резерв & -- & -- & -- & uint32 \tabularnewline\hline
	Резерв & -- & -- & -- & uint32 \tabularnewline\hline
	Резерв & -- & -- & -- & uint32 \tabularnewline\hline
	Резерв & -- & -- & -- & uint32 \tabularnewline\hline
	Резерв & -- & -- & -- & uint32 \tabularnewline\hline
	Резерв & -- & -- & -- & uint32 \tabularnewline\hline
	Резерв & -- & -- & -- & uint32 \tabularnewline\hline
	Резерв & -- & -- & -- & uint32 \tabularnewline\hline
	Резерв & -- & -- & -- & uint32 \tabularnewline\hline
	Резерв & -- & -- & -- & uint32 \tabularnewline\hline
	Резерв & -- & -- & -- & uint32 \tabularnewline\hline
	Резерв & -- & -- & -- & uint32 \tabularnewline\hline
	Резерв & -- & -- & -- & uint32 \tabularnewline\hline
	Резерв & -- & -- & -- & uint32 \tabularnewline\hline
	Резерв & -- & -- & -- & uint32 \tabularnewline\hline
	%
%	\multicolumn{5}{|l|}%
%	{%
%		\setstretch{0.7}%
%		%		\hspace{-1mm}% Добавление абзацного отступа (откуда он взялся - хз)
%		\tabletextsize%
%		\note Размер структуры N байт.
%	} \tabularnewline\hline
\end{longtable}
}

\newpage
Выходные данные комплекса состоят из следующих структур данных:
\begin{enumerate}
	\item[--] заголовок;
	\item[--] выходные данные;
	\item[--] массив ка-цэ.
\end{enumerate}

Структуры данных приведены в таблицах ниже.
%%%%%%%%%%%%%%%%%%%%%%%%%%%%%%%%%%%%%%%%%%%%%%%%%%%%%%%%%%%%%%%%%%%%%%%%%%%%%%%
%
% Структура данных
%
%\subsubsection{Выходные данные }
%\vspace{0.5cm}
{\tabletextsize
\begin{longtable}[c]{| >{\raggedright}m{\wtname} | >{\centering}m{\wtsymbol} | >{\centering}m{\wtunits} | >{\centering}m{\wtbounds} | >{\centering}m{\wtcomment} |}
		%----------------------- преамбула ---------------------	
		\caption{\normalsize Структура выходных данных\hspace{25cm}}
		\label{t:Выходные_данные} \\
		\hline
		\centering{Наименование информации} & 
		\centering{Условное\\обозначение} & 
		\centering{Размер-\\ность} & 
		\centering{Пределы\\изменения} & 
		\centering{Примеча-\\ние} \tabularnewline
		\hhline{|=|=|=|=|=|}
		\endfirsthead % Конец заголовка на 1 странице
		\multicolumn{5}{l}{\textit{Продолжение таблицы \thetable}} \\ 
		\hline	
		% Способ, при котором раздельно формуруется выравнивание заголовка и контетнта
		\centering{Наименование информации} & 
		\centering{Условное\\обозначение} & 
		\centering{Размер-\\ность} & 
		\centering{Пределы\\изменения} & 
		\centering{Примеча-\\ние} \tabularnewline
		\hhline{|=|=|=|=|=|}
		\endhead
		\hline
		%		\multicolumn{5}{r}{\tabletextsize см. далее}
		\endfoot
		\hline
		\endlastfoot	
		%------------------- табличные данные ------------------		
		Время привязки пацаков & $Time$ & c & $0$\mbdash$(2^{32}-1)$ & uint32 \tabularnewline\hline		
		%		
%		\multicolumn{5}{|l|}%
%		{%
%			\setstretch{0.7}%
%			%		\hspace{-1mm}% Добавление абзацного отступа (откуда он взялся - хз)
%			\tabletextsize%
%			\note Размер структуры 36 байт.
%		} \tabularnewline\hline
	\end{longtable}
}

%%%%%%%%%%%%%%%%%%%%%%%%%%%%%%%%%%%%%%%%%%%%%%%%%%%%%%%%%%%%%%%%%%%%%%%%%%%%%%%%%
%
% Структура ка-цэ
%
\newpage
{\tabletextsize
\begin{longtable}[c]{| >{\raggedright}m{\wtname} | >{\centering}m{\wtsymbol} | >{\centering}m{\wtunits} | >{\centering}m{\wtbounds} | >{\centering}m{\wtcomment} |}
		%----------------------- преамбула ---------------------	
		\caption{\normalsize Структура ка-цэ\hspace{25cm}}
		\label{t:Выходные_данные_ка-цэ} \\
		\hline
		\centering{Наименование информации} & 
		\centering{Условное\\обозначение} & 
		\centering{Размер-\\ность} & 
		\centering{Пределы\\изменения} & 
		\centering{Примеча-\\ние} \tabularnewline
		\hhline{|=|=|=|=|=|}
		\endfirsthead % Конец заголовка на 1 странице
		\multicolumn{5}{l}{\textit{Продолжение таблицы \thetable}} \\ 
		\hline	
		% Способ, при котором раздельно формуруется выравнивание заголовка и контетнта
		\centering{Наименование информации} & 
		\centering{Условное\\обозначение} & 
		\centering{Размер-\\ность} & 
		\centering{Пределы\\изменения} & 
		\centering{Примеча-\\ние} \tabularnewline
		\hhline{|=|=|=|=|=|}
		\endhead
		\hline
		%		\multicolumn{5}{r}{\tabletextsize см. далее}
		\endfoot
		\hline
		\endlastfoot	
		%------------------- табличные данные ------------------		
		Номер ка-цэ & $N$ & б/р & $1$\mbdash$(2^{32}-1)$ & uint32 \tabularnewline\hline		
		Резерв & -- & -- & -- & uint32 \tabularnewline\hline
		Резерв & -- & -- & -- & uint32 \tabularnewline\hline
		Резерв & -- & -- & -- & uint32 \tabularnewline\hline
		Резерв & -- & -- & -- & uint32 \tabularnewline\hline
		Резерв & -- & -- & -- & uint32 \tabularnewline\hline
		Резерв & -- & -- & -- & uint32 \tabularnewline\hline
		Резерв & -- & -- & -- & uint32 \tabularnewline\hline
		Резерв & -- & -- & -- & uint32 \tabularnewline\hline
		Резерв & -- & -- & -- & uint32 \tabularnewline\hline
		Резерв & -- & -- & -- & uint32 \tabularnewline\hline
		Резерв & -- & -- & -- & uint32 \tabularnewline\hline
		Резерв & -- & -- & -- & uint32 \tabularnewline\hline
		%
%		\multicolumn{5}{|l|}%
%		{%
%			\setstretch{0.7}%
%			%		\hspace{-1mm}% Добавление абзацного отступа (откуда он взялся - хз)
%			\tabletextsize%
%			\note Размер структуры N байт.
%		} \tabularnewline\hline
	\end{longtable}
}


\subsection{Состав технических и программных средств} 

Комплекс реализуется на вычислительных средствах изделия \productname{} и выполняет свои функции только в процессе совместной работы с другими комплексами и аппаратурой изделия при поступлении корректной и непротиворечивой входной информации.

Перечень функциональных объектов, образующих программу комплекса приведен в приложении к настоящему документу.

Обмен информацией между комплексом и другими комплексами из состава изделия, информационно сопряженными с ним, осуществляется с использованием соответствующих протоколов обмена.

Использование комплекса программ на других вычислительных средствах не предусматривается.
